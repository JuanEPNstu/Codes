\documentclass[12pt]{article}
\usepackage[spanish]{babel}
\usepackage{geometry}
\geometry{a4paper, margin=1in}
\usepackage{graphicx}
\usepackage{xcolor}
\usepackage{titlesec}
\usepackage{parskip}
\usepackage{multicol}
\usepackage{cite}
\usepackage{float}

\definecolor{highlight}{RGB}{255, 255, 0}

\titleformat{\section}{\normalfont\Large\bfseries}{\thesection}{1em}{}
\titleformat{\subsection}{\normalfont\large\bfseries}{\thesubsection}{1em}{}

\begin{document}

% Logos
\begin{minipage}{0.45\textwidth}
    \includegraphics[width=0.4\textwidth]{inFiles/Figures/epnLogo.jpg}
\end{minipage}
\hfill
\begin{minipage}{0.45\textwidth}
    \raggedleft
    \includegraphics[width=0.4\textwidth]{inFiles/Figures/FIS_logo.jpg}
\end{minipage}

\vspace{0.5cm}

% Títulos principales
\begin{center}
    \textbf{ESCUELA POLITÉCNICA NACIONAL}\\[0.2cm]
    \textbf{FACULTAD DE INGENIERÍA DE SISTEMAS}\\[0.2cm]
    \textbf{INGENIERÍA {\textbf{EN COMPUTACIÓN}}}
\end{center}

\vspace{0.5cm}
\hrule
\vspace{0.5cm}

% Datos principales
\noindent\textbf{PERÍODO ACADÉMICO:} 2025-A\\[0.2cm]
\noindent\textbf{ASIGNATURA:} ICCD412 Métodos Numéricos \hfill \textbf{GRUPO:} GR2\\[0.2cm]
\noindent\textbf{TIPO DE INSTRUMENTO:} Tarea 10\\[0.2cm]
\noindent\textbf{FECHA DE ENTREGA LÍMITE:} 01/06/2025\\[0.2cm]
\noindent\textbf{ALUMNO:} Murillo Tobar Juan

\vspace{0.5cm}
\hrule
\vspace{1cm}


% Secciones
\section*{TEMA}
Splines Cúbicos

\vspace{0.5cm}

\section*{OBJETIVOS}
\begin{itemize}
    \item Comprender como se obtiene las ecuaciones para hallar los coeficientes para los diferentes splines cúbicos.
    \item Realizar ejercicios con splines en condiciones frontera natural y condicionado, además saber cuales son sus diferencias.

\end{itemize}

\vspace{0.5cm}


\section*{DESARROLLO}

\textbf{Dados los puntos x = [-2, - 1, 1,3], y = [3, 1, 2, -1]}
\textbf{a) Determine el spline cúbico con frontera natural}



Primero establecemos los intervalos para los splines: $[-2,-1]$, $[-1,1]$, $[1,3]$

$S_0 = a_0+b_0(x+2)+c_0(x+2)^2+d_0(x+2)^3 = y_0 = 3$

$S_1 = a_1+b_1(x+1)+c_1(x+1)^2+d_1(x+1)^3 = y_1 = 1$

$S_2 = a_2+b_2(x-1)+c_2(x-1)^2+d_2(x-1)^3 = y_2 = 2$


Ahora encontramos las ecuaciones para obtener los coeficientes.

$a_0+b_0(0)+c_0(0)^2+d_0(0)^3 = y_0 = 3$

\textcolor{red}{(1)} $a_0 = 3$

$a_0+b_0+c_0+d_0 = 1$

\textcolor{red}{(2)} $b_0+c_0+d_0 = -2$

$a_1+b_1(0)+c_1(0)+d_1(0) = 1$

\textcolor{red}{(3)} $a_1 = 1$

$a_1+2b_1+4c_1+8d_1 = 2$

\textcolor{red}{(4)} $2b_1+4c_1+8d_1 = 1$

\textcolor{red}{(5)} $a_2 = 2$

$a_2+2b_2+4c_2+8d_2 = -1$

\textcolor{red}{(6)}  $2b_2+4c_2+8d_2 = -3$

$b_0 +2c_0(x_1+2)+3d_0(x_1+2)^2 = b_1+2c_1(0)+3d_1(0)$

\textcolor{red}{(7)}  $b_0+2c_0+3d_0=b_1$

$2c_0+6d_0(x_1+2) = 2c_1+6d_1(0)$

\textcolor{red}{(8)} $2c_0+6d_0=2c_1$

$b_1 +2c_1(2)+3d_1(2)^2 = b_2$

\textcolor{red}{(9)} $b_1+4c_1+12d_1=b_2$

\textcolor{red}{(10)} $2c_1+12d_1=2c_2$

\textcolor{red}{(11)} $2c_0=0$

\textcolor{red}{(12)} $2c_2+12d_2=0$

\textcolor{red}{(11.a)} $b_0=B_0=1$

\textcolor{red}{(12.b)} $b_2+4c_2+12d_2=-1$

Resolvemos el sistema de ecuaciones y obtenemos

$d_1 = \frac{1}{8} - \frac{1}{2}c_1- \frac{1}{4}b_1$


$b_2 = -2b_1 - 2c_1 + \frac{3}{2}$

$2c_2 = -4c_1 - 3b_1 + \frac{3}{2}$

$2b_2 + \frac{8}{3}c_2 = -3$

$-8b_1-\frac{28}{3}c_1=-8$

Reemplazando con las demás ecuaciones obtenemos los siguiente splines

$S_0 =  3-\frac{28}{3}(x+2)+\frac{6}{11}(x+2)^3 = 3$

$S_1 = 1-\frac{10}{11}(x+1)+\frac{18}{11}(x+1)^2-\frac{41}{88}(x+1)^3 = 1$

$S_2 = 2+\frac{1}{22}(x-1)-\frac{51}{44}(x-1)^2+\frac{17}{88}(x-1)^3 = 2$

\small
$6 -\frac{28}{3}(x+2)+\frac{6}{11}(x+2)^3 -\frac{10}{11}(x+1)+\frac{18}{11}(x+1)^2-\frac{41}{88}(x+1)^3+\frac{1}{22}(x-1)-\frac{51}{44}(x-1)^2+\frac{17}{88}(x-1)^3 = 6$
\normalsize

\textbf{b) Determine el spline cúbico con frontera condicionada B0 = 1 BN=-1}

En este caso obtendremos los siguientes splines

$S_0 =  3-1(x+2)-\frac{136}{23}(x+2)^2+\frac{67}{23}(x+2)^3 = 3$

$S_1 = 1-\frac{48}{23}(x+1)+\frac{65}{23}(x+1)^2-\frac{141}{184}(x+1)^3 = 1$

$S_2 = 2+\frac{1}{46}(x-1)-\frac{163}{92}(x-1)^2+\frac{93}{184}(x-1)^3 = 2$


\textbf{Dados los puntos (0,1); (1,5); (2,3), determine el spline cúbico}

Primero establecemos los intervalos para los splines: $[0,1]$, $[1,2]$

$S_0 = a_0+b_0(x-0)+c_0(x-0)^2+d_0(x-0)^3 = y_0 = 1$

$S_1 = a_1+b_1(x-1)+c_1(x-1)^2+d_1(x-1)^3 = y_1 = 5$

Obtenemos las siguientes ecuaciones


\textcolor{red}{(1)} $a_0 = 1$

\textcolor{red}{(2)} $b_0+c_0+d_0 = 4$

\textcolor{red}{(3)} $a_1 = 5$

\textcolor{red}{(4)} $b_1+c_1+d_1 = -2$

\textcolor{red}{(5)} $b_0+2c_0+3d_0=b_1$

\textcolor{red}{(6)} $2c_0+6d_0=2c_1$

\textcolor{red}{(7)} $2c_0=0$

\textcolor{red}{(8)} $2c_1+6d_1=0$

Que resolviendo el sistema obtenemos los siguientes splines

$S_0 = 1+\frac{11}{2}(x-0)-\frac{3}{2}(x-0)^3 = 1$

$S_1 = 5+1(x-1)-\frac{9}{2}(x-1)^2+\frac{11}{2}(x-1)^3 = 5$



\textbf{Dados los puntos (-1,1); (1,3);(0.5,4.8), determine el spline cúbico sabiendo que  $f'(x_0) = 1$, $f' (x_n) = 2$}

Primero establecemos los intervalos para los splines: $[-1,1]$, $[1,0.5]$

$S_0 = a_0+b_0(x+1)+c_0(x+1)^2+d_0(x+1)^3 = y_0 = 1$

$S_1 = a_1+b_1(x-1)+c_1(x-1)^2+d_1(x-1)^3 = y_1 = 3$

Obtenemos las siguientes ecuaciones


\textcolor{red}{(1)} $a_0 = 1$

\textcolor{red}{(2)} $2b_0+4c_0+8d_0 = 2$

\textcolor{red}{(3)} $a_1 = 3$

\textcolor{red}{(4)} $-0.5b_1+0.25c_1-0.125d_1 = 1.8$

\textcolor{red}{(5)} $b_0+4c_0+12d_0=b_1$

\textcolor{red}{(6)} $2c_0+12d_0=2c_1$

\textcolor{red}{(7)} $b_0=1$

\textcolor{red}{(8)} $b_1-c_1+0.75d_1=2$

En este caso para obtener mas precisión realizaremos las operaciones en forma de fracción. 

$S_0 = 1+1(x+1)+\frac{74}{15}(x+1)^2-\frac{37}{15}(x+1)^3 =1$

$S_1 = 3-\frac{133}{15}(x-1)-\frac{148}{15}(x-1)^2+\frac{4}{3}(x-1)^3 = 3$

\vspace{0.5cm}


\renewcommand{\refname}{\MakeUppercase{REFERENCIAS}}
\bibliographystyle{IEEEtran}
\bibliography{inFiles/References/references.bib}

\end{document}
