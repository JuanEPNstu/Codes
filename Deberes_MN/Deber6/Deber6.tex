\documentclass[12pt]{article}
\usepackage[spanish]{babel}
\usepackage{geometry}
\geometry{a4paper, margin=1in}
\usepackage{graphicx}
\usepackage{xcolor}
\usepackage{titlesec}
\usepackage{parskip}
\usepackage{multicol}
\usepackage{cite}

\definecolor{highlight}{RGB}{255, 255, 0}

\titleformat{\section}{\normalfont\Large\bfseries}{\thesection}{1em}{}
\titleformat{\subsection}{\normalfont\large\bfseries}{\thesubsection}{1em}{}

\begin{document}

% Logos
\begin{minipage}{0.45\textwidth}
    \includegraphics[width=0.4\textwidth]{inFiles/Figures/epnLogo.jpg}
\end{minipage}
\hfill
\begin{minipage}{0.45\textwidth}
    \raggedleft
    \includegraphics[width=0.4\textwidth]{inFiles/Figures/FIS_logo.jpg}
\end{minipage}

\vspace{0.5cm}

% Títulos principales
\begin{center}
    \textbf{ESCUELA POLITÉCNICA NACIONAL}\\[0.2cm]
    \textbf{FACULTAD DE INGENIERÍA DE SISTEMAS}\\[0.2cm]
    \textbf{INGENIERÍA{\textbf{ EN COMPUTACIÓN}}}
\end{center}

\vspace{0.5cm}
\hrule
\vspace{0.5cm}

% Datos principales
\noindent\textbf{PERÍODO ACADÉMICO:} 2025-A\\[0.2cm]
\noindent\textbf{ASIGNATURA:} ICCD412 Métodos Numéricos \hfill \textbf{GRUPO:} GR2\\[0.2cm]
\noindent\textbf{TIPO DE INSTRUMENTO:} Tarea 6\\[0.2cm]
\noindent\textbf{FECHA DE ENTREGA LÍMITE:} 09/05/2025\\[0.2cm]
\noindent\textbf{ALUMNO:} Murillo Tobar Juan 

\vspace{0.5cm}
\hrule
\vspace{1cm}


% Secciones
\section*{TEMA}
Método de la secante

\vspace{0.5cm}

\section*{OBJETIVOS}
\begin{itemize}
    \item Utilizar el método de secante y de Newton para realizar comparaciones en base a sus iteraciones.
    \item Comprender porque el método de secante es mejor en comparación con el método de Newton Raphson en términos de complejidad en sus operaciones.
\end{itemize}

\vspace{0.5cm}

\section*{MARCO TEÓRICO}
\large\textbf{Método de la secante}
\normalsize

Como se menciona en \cite{sauer2013} el método de la secante es muy similar al método de Newton con la diferencia que en este método se utilizan dos aproximaciones iniciales y que la derivada es reemplazada con un cociente de diferencias finitas. 
Por lo anterior mencionado podemos decir que la complejidad en las operaciones es menor y por lo tanto el tiempo de computo también va a ser menor en el método de la secante, pero hay que aclarar que ambos no dejan de ser métodos abiertos con posibilidad a divergencia.

\vspace{0.5cm}

\section*{DESARROLLO}
1. Use el método de la secante para encontrar una solución para
 $$x = \cos(x) (f(x) = \cos(x) - x = 0)$$ 
con tolerancia tal que:
$$
\left| P_n - P_{n-1} \right| < (tolerancia = 10^{-16})
$$

y compare las aproximaciones con las determinadas en el ejemplo visto en clase, el cual aplica el método
de Newton, resuelva hasta llegar a la misma tolerancia para este método también.
Suponga que usamos $p_0 = 0.5$ y $p_1 = \frac{\pi}{4}$ , trabaje con 13 cifras decimales de redondeo.

\large\textbf{Fórmula Secante}
\normalsize

$$
X_n = X_{n-1} - f(X_{n-1})\times \frac{X_{n-1} - X_{n-2}}{f(X_{n-1}) - f(X_{n-2})}
$$


\large\textbf{Aproximaciones por Secante}
\small

\begin{center}
    \begin{tabular}{|c|c|c|c|c|}
        \hline
        $X_n$ & $X_{n-1}$ & $X_{n-2}$ & $f(X_{n})$ & $E_est$\\
        \hline
        $0.5000000000000$ & $ $               & $ $               & $0.3775825619000$ & $ $\\
        $0.7853981634000$ & $0.5000000000000$ & $ $               & $-0.0782913822100$ & $0.2854$\\
        $0.7363841388000$ & $0.7853981634000$ & $0.5000000000000$ & $0.0045177185830$ & $0.04901$\\
        $0.7390581564000$ & $0.7363841388000$ & $0.7853981634000$ & $0.0000451484193$ & $2.674*10^{-3}$\\
        $0.7390851493000$ & $0.7390581564000$ & $0.7363841388000$ & $-0.269198*10^{-7}$ & $2.699*10^{-5}$\\
        $0.7390851332000$ & $0.7390851493000$ & $0.7390581564000$ & $0.254*10^{-10} $ & $1.61*10^{-8}$\\
        \hline
      \end{tabular} 
\end{center}

\large\textbf{Fórmula Newton}
\normalsize

$$
X_n = X_{n-1} - \frac{f(X_{n-1})}{f'(X_{n-1})}, n\geq 1
$$

\large\textbf{Aproximaciones por Newton}
\normalsize

\begin{center}
    \begin{tabular}{|c|c|c|c|c|}
        \hline
        $X_n$ & $X_{n-1}$& $E_est$\\
        \hline
        $0.7395361335000$ & $\frac{\pi}{4} $& $0.04586$\\
        $0.7390851781000$ & $0.7395361335000$ & $4.51*10^{-4}$\\
        $0.7390851332000$ & $0.7390851781000$ & $4.49*10^{-8}$\\

        \hline
      \end{tabular} 
\end{center}








\vspace{0.5cm}


\renewcommand{\refname}{\MakeUppercase{REFERENCIAS}}
\bibliographystyle{IEEEtran}
\bibliography{inFiles/References/references.bib}


\end{document}
