\documentclass[12pt]{article}
\usepackage[spanish]{babel}
\usepackage{geometry}
\geometry{a4paper, margin=1in}
\usepackage{graphicx}
\usepackage{xcolor}
\usepackage{titlesec}
\usepackage{parskip}
\usepackage{multicol}
\usepackage{cite}
\usepackage{float}

\definecolor{highlight}{RGB}{255, 255, 0}

\titleformat{\section}{\normalfont\Large\bfseries}{\thesection}{1em}{}
\titleformat{\subsection}{\normalfont\large\bfseries}{\thesubsection}{1em}{}

\begin{document}

% Logos
\begin{minipage}{0.45\textwidth}
    \includegraphics[width=0.4\textwidth]{inFiles/Figures/epnLogo.jpg}
\end{minipage}
\hfill
\begin{minipage}{0.45\textwidth}
    \raggedleft
    \includegraphics[width=0.4\textwidth]{inFiles/Figures/FIS_logo.jpg}
\end{minipage}

\vspace{0.5cm}

% Títulos principales
\begin{center}
    \textbf{ESCUELA POLITÉCNICA NACIONAL}\\[0.2cm]
    \textbf{FACULTAD DE INGENIERÍA DE SISTEMAS}\\[0.2cm]
    \textbf{INGENIERÍA {\textbf{EN COMPUTACIÓN}}}
\end{center}

\vspace{0.5cm}
\hrule
\vspace{0.5cm}

% Datos principales
\noindent\textbf{PERÍODO ACADÉMICO:} 2025-A\\[0.2cm]
\noindent\textbf{ASIGNATURA:} ICCD412 Métodos Numéricos \hfill \textbf{GRUPO:} GR2\\[0.2cm]
\noindent\textbf{TIPO DE INSTRUMENTO:} Tarea 9\\[0.2cm]
\noindent\textbf{FECHA DE ENTREGA LÍMITE:} 25/05/2025\\[0.2cm]
\noindent\textbf{ALUMNO:} Murillo Tobar Juan

\vspace{0.5cm}
\hrule
\vspace{1cm}


% Secciones
\section*{TEMA}
Polinomio de Taylor

\vspace{0.5cm}

\section*{OBJETIVOS}
\begin{itemize}
    \item Comprender como se obtiene el polinomio de Taylor a partir de la resolución y simplificación de sumatorios que implican multiplicaciones.
    \item Reconocer las partes que componen a la función de aproximación, es decir, el polinomio de Taylor y el error de Truncamiento.

\end{itemize}

\vspace{0.5cm}

\section*{MARCO TEÓRICO}

\textbf{Polinomio de Taylor}
\normalsize\newline\newline
Como se menciona en \cite{book:2450406}, este método nos proporciona un medio para aproximar 
un valor de una función en un punto en términos del  valor de la función y sus derivadas en otro punto.
Además, se establece que cualquier "smooth function", es decir una función que es derivable infinitamente, puede ser aproximada como un polinomio. 
\vspace{0.5cm}

\section*{DESARROLLO}

\textbf{Dada la función $e^{-x}$  y $x_0 = 1$. Determinar la función aproximada $f(x)=P_n (x)+R_n(x)$ con n = 3.}

Primero sacamos todas sus derivada, es decir hasta de orden 4, ya que el error de truncamiento usa 1 mas que el polinomio.
$f(x) = e^{-x}$

$f'(x) = -e^{-x}$

$f''(x) = e^{-x}$

$f'''(x) = -e^{-x}$

$f''''(x) = e^{-x}$

Ahora comenzamos a resolver el sumatorio y nos quedaría la siguiente forma

$$P_n(x) = e^{-1}-e^{-1}(x-1)+ \frac{e^{-1}}{2}(x-1)^2-\frac{e^{-1}}{6}(x-1)^3$$

Al elaborar un poco mas nos queda

$$P_n(x) =  e^{-1} - e^{-1}x + e^{-1} + \frac{e^{-1}}{2}x^2-e^{-1}x+\frac{e^{-1}}{2}-\frac{e^{-1}}{6}x^3+\frac{e^{-1}}{2}x^2-\frac{e^{-1}}{2}x+\frac{e^{-1}}{6}$$

Al simplificar términos semejantes nos queda.

$$P_n(x) = \frac{-1}{6e}x^3+\frac{1}{e}x^2-\frac{5}{2e}x+\frac{16}{6e}$$

Ahora resolvemos el error de truncamiento Rn, aunque en este lo dejamos expresado tal como esta


$$R_n = \frac{e^{\xi(x)}}{24}(x-1)^4$$

$$R_n = \frac{e^{\xi(x)}}{24}(x^4-4x^3+6x^2-4x+1)$$

La función de aproximación nos quedaría como 

$$P_n(x) = \frac{-1}{6e}x^3+\frac{1}{e}x^2-\frac{5}{2e}x+\frac{16}{6e} + \frac{e^{\xi(x)}}{24}(x^4-4x^3+6x^2-4x+1)$$



\vspace{0.5cm}


\renewcommand{\refname}{\MakeUppercase{REFERENCIAS}}
\bibliographystyle{IEEEtran}
\bibliography{inFiles/References/references.bib}

\end{document}
