\documentclass[12pt]{article}
\usepackage[spanish]{babel}
\usepackage{geometry}
\geometry{a4paper, margin=1in}
\usepackage{graphicx}
\usepackage{xcolor}
\usepackage{titlesec}
\usepackage{parskip}
\usepackage{multicol}
\usepackage{cite}

\definecolor{highlight}{RGB}{255, 255, 0}

\titleformat{\section}{\normalfont\Large\bfseries}{\thesection}{1em}{}
\titleformat{\subsection}{\normalfont\large\bfseries}{\thesubsection}{1em}{}

\begin{document}

% Logos
\begin{minipage}{0.45\textwidth}
    \includegraphics[width=0.4\textwidth]{inFiles/Figures/epnLogo.jpg}
\end{minipage}
\hfill
\begin{minipage}{0.45\textwidth}
    \raggedleft
    \includegraphics[width=0.4\textwidth]{inFiles/Figures/FIS_logo.jpg}
\end{minipage}

\vspace{0.5cm}

% Títulos principales
\begin{center}
    \textbf{ESCUELA POLITÉCNICA NACIONAL}\\[0.2cm]
    \textbf{FACULTAD DE INGENIERÍA DE SISTEMAS}\\[0.2cm]
    \textbf{INGENIERÍA{\textbf{ EN COMPUTACIÓN}}}
\end{center}

\vspace{0.5cm}
\hrule
\vspace{0.5cm}

% Datos principales
\noindent\textbf{PERÍODO ACADÉMICO:} 2025-A\\[0.2cm]
\noindent\textbf{ASIGNATURA:} ICCD412 Métodos Numéricos \hfill \textbf{GRUPO:} GR2\\[0.2cm]
\noindent\textbf{TIPO DE INSTRUMENTO:} Tarea 2\\[0.2cm]
\noindent\textbf{FECHA DE ENTREGA LÍMITE:} 04/05/2025\\[0.2cm]
\noindent\textbf{ALUMNO:} Murillo Tobar Juan 

\vspace{0.5cm}
\hrule
\vspace{1cm}


% Secciones
\section*{TEMA}
Cálculo de error

\vspace{0.5cm}

\section*{OBJETIVOS}
\begin{itemize}
    \item Practicar el cálculo de los cuatro tipo de errores (real, absoluto, relativo, relativo porcentual) para futuros problemas dentro de la asignatura.
    \item Reconocer las diferencias entre el error de truncamiento y el error de redondeo, y aprender cómo calcular ambos.
\end{itemize}

\vspace{0.5cm}

\section*{MARCO TEÓRICO}
\colorbox{highlight}{No solicitado} 

\vspace{0.5cm}

\section*{DESARROLLO}
Tomamos la constante $\pi$  para realizar truncamiento y redondeo a 4 cifras significativas. Luego debíamos obtener los resultados de los 4 errores.

\large\textbf{Con redondeo}
\begin{itemize}
\item Error Real
$$\pi - 3.142 = -0.0004073$$
\item Error Absoluto
$$
\left| \pi - 3.142 \right| = 0.0004073
$$
\item Error Relativo
$$
\left| \frac{\pi - 3.142}{\pi} \right| = 0.0001297
$$
\item Error Relativo Porcentual
$$
\left| \frac{\pi - 3.142}{\pi} *100\% \right| = 0.01297\%
$$ 
\end{itemize}
\large\textbf{Con truncamiento}
\begin{itemize}
    \item Error Real
    $$\pi - 3.141 = 0.0005926$$
    \item Error Absoluto
    $$
    \left| \pi - 3.141 \right| = 0.0005926
    $$
    \item Error Relativo
    $$
    \left| \frac{\pi - 3.141}{\pi} \right| = 0.0001886
    $$
    \item Error Relativo Porcentual
    $$
    \left| \frac{\pi - 3.141}{\pi} *100\% \right| = 0.01886\%
    $$ 
    \end{itemize}
\vspace{0.5cm}


\renewcommand{\refname}{\MakeUppercase{REFERENCIAS}}
%\bibliographystyle{IEEEtran}
%\bibliography{inFiles/References/references.bib}


\end{document}
